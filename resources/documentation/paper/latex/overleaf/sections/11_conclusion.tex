\section{Conclusion}

As large language models enable increasingly sophisticated multi-step reasoning, the problem of context pollution threatens to undermine their reliability. When information accumulates implicitly across reasoning steps, models hallucinate, confuse intermediate outputs, and lose track of constraints---making complex workflows paradoxically more brittle than simple ones.

NormCode addresses this through \textbf{enforced data isolation}: each inference operates in a sealed environment with only explicitly passed inputs. This is not a convention but a language-level constraint that the compiler and orchestrator enforce. Combined with a clean semantic/syntactic separation (probabilistic LLM reasoning vs. deterministic data flow), NormCode makes AI workflows auditable by construction.

The multi-format ecosystem (\texttt{.ncds} for authoring, \texttt{.ncd} for execution, \texttt{.ncn} for verification, \texttt{.ncdn} for hybrid editing) embodies a broader principle: transparent AI systems should provide multiple views of the same underlying logic, each optimized for different stakeholders. Domain experts can verify plans in natural language; developers can debug in formal syntax; auditors can trace decisions with precision; and the orchestrator can execute with formal rigor.

The four-phase compilation pipeline (Derivation, Formalization, Post-Formalization, Activation) enables progressive formalization where plans tighten incrementally from rough sketches to production-ready specifications. Each phase answers specific questions while preserving semantic intent, and manual review opportunities allow human intervention at critical junctures.

The Canvas App provides visual debugging capabilities that transform NormCode from ``run and hope'' to ``observe and control.'' Real-time graph visualization, breakpoint debugging, tensor inspection, and multi-agent configuration make the execution process transparent and manipulable.

We validate the approach through self-hosted execution (NormCode's compiler runs as a NormCode plan) and algorithmic correctness (100\% accuracy on base-X addition). These demonstrations confirm that structured intermediate representations can bridge human intuition and machine execution without sacrificing either.

NormCode is not appropriate for every use case---simple tasks do not justify the overhead, and real-time applications may not tolerate orchestration latency. But for high-stakes domains where failures have consequences (legal, medical, financial), the ability to answer ``What did step 7 actually see?'' is not optional. By making context isolation structural rather than aspirational, NormCode provides a foundation for AI systems that are reliable, transparent, and auditable---requirements that will only intensify as LLMs move from research prototypes into production systems.

